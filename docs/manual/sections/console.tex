\section{The Open Chord Console}\label{sectionconsole}

For testing purposes Open Chord provides a console, from which a network can be 
created. For this console a {\tt Key} implementation based on {\tt String} similar 
to that described in the previous section is being used. Instances of
{\tt String} are used as data. There are two 
possibilities two create/join an Open Chord network. The first one uses the 
{\tt oclocal} protocol to create a network within one JVM. Therefore many local peers 
can be created with help of the console. The other possibility is to instantiate 
a single peer that uses {\tt ocsocket} protocol to create or join a remote 
Open Chord network. 

In order to start the console the following command has to be executed:  
\begin{verbatim}
java -cp OC_DIR/build/classes;OC_DIR/config;OC_DIR/lib/log4j.jar 
de.uniba.wiai.lspi.chord.console.Main
\end{verbatim}
{\tt OC$\_$DIR} is the directory where the directory structure of Open Chord 
as described in section \ref{installation} has been created. If console 
execution is successful, your screen looks as presented in figure \ref{console}. 
Note that the reference to {\tt log4j} in the classpath ({\tt cp}) is optional 
as {\tt log4j} is not required to use Open Chord.  

\bildw{console}{console}{Open Chord console.}{16}

When the console is running it waits for commands to be typed in. There are 
two kinds of commands. The first kind of commands is designed to be used 
with an Open Chord network that runs within one JVM. The second kind of 
commands is designed to be used with a single peer, that is connected to 
other remote peers to create a Open Chord network. 
The next sections (\ref{localcom}, \ref{remotecom}) describe these 
commands and how they can be used. 

There are also some other commands available from the console. 
To get a list of available commands from the console type {\tt help} at 
the prompt and press enter. Every listed command also provides help 
to the user, when it is issued with the parameter {\tt -h} or {\tt -help}. 
To close the console type {\tt exit}. 
In order to prevent the console from becoming unreadable if classes use 
{\tt System.out}, {\tt System.out} is redirected and the output is saved in 
memory. To display this output type {\tt displaySystemOut}. Output of 
{\tt System.err} is still printed to the console. 
Some times it may be useful to repeatedly execute the same commands in the same 
order. For this purpose a macro can be created. This macro is a simple text file, 
which contains a command in each line. This macro file can be loaded and executed 
with help of the {\tt executeMacro} command. 

\subsection{Simulating a Chord overlay network in one JVM}\label{localcom}
With the {\tt oclocal} communication protocol for Open Chord it is possible 
to create an Open Chord network within one JVM. For this purpose the console 
provides commands, that can be used to e.g. create a network, insert data, or 
crash a node. As most of the commands affect only one peer the name of this peer 
must be provided to these commands. Names of peers have to be unique within 
the JVM and an {\tt URL} of a peer, that uses the {\tt oclocal} protocol, 
contains the name of the peer as the host part of the {\tt URL} (e.g. {\tt oclocal://mypeer/}. 

These commands are: 
\begin{itemize}
%
\item {\tt create}: A new network can be created or 
an existing network can be joined with this command. 
To create a new network this command 
expects one name of a peer provided with help of the {\tt names} parameter
(e.g. {\tt create -names mypeer}). If a network exists, this command can 
be used to create a number of peers at once. The names of the peers must 
be concatenated with help of '{\tt $\_$}'. In addition to the names 
a list of bootstrap peers is required. These list must at least contain 
one name of an existing peer. If there is more than one bootstrap peer, 
the names must again be separated by '{\tt $\_$}'. 

{\bf Example:}\footnote{If the examples in this section are processed in the 
same order as in this manual, their output will look similar to the examples!}
\begin{verbatim}
oc > create -names mypeer0 
Creating new chord network. 
oc > create -names mypeer1_mypeer2 -bootstraps mypeer0
Starting node with name 'mypeer1' with bootstrap node 'mypeer0'
Starting node with name 'mypeer2' with bootstrap node 'mypeer0'
oc > create -names mypeer3_mypeer4_mypeer5 -bootstraps mypeer0_mypeer1
Starting node with name 'mypeer3' with bootstrap node 'mypeer0'
Starting node with name 'mypeer4' with bootstrap node 'mypeer1'
Starting node with name 'mypeer5' with bootstrap node 'mypeer1'
oc > 
\end{verbatim}
First a new network with a single peer with name {\tt mypeer0} is created. 
Afterward two peers, that use both {\tt mypeer0} as bootstrap node are simultaneously 
created. Then three peers are created. The first one uses {\tt mypeer0} as bootstrap 
peer and the last two use {\tt mypeer1} as bootstrap peer. Bootstrap peers from 
the list are assigned to the peers at the same position of the list of peers to create. 
If there are fewer bootstrap peers than peers to create, the last bootstrap peer 
of the bootstrap peer list is assigned to all peers at the end of the list as a 
bootstrap peer. 
%
\item {\tt insert}: This command can be used to insert a {\tt String} with a 
certain key into the Chord DHT. This command expects three parameters: {\tt node}, 
{\tt key}, and {\tt value}. The first one is the name of the peer, from which the 
insert request is issued. The second parameter is the key, that is used to store 
the {\tt String}, which is provided with help of the third parameter. 

{\bf Example:}
\begin{verbatim}
oc > insert -node mypeer1 -key test -value test
Value 'test' with key 'test' inserted successfully from node 'mypeer1'.
oc >
\end{verbatim}
%
\item {\tt entries}: This command can be used to print the data and the associated 
keys stored by all peers in the network. If the name of the peer is provided with 
help of parameter {\tt node} the entries of this peer are printed to the console. 

{\bf Example:}
\begin{verbatim}
oc > entries
Node mypeer5: Entries:
  key = A9 , value = [( key = A9 , value = test)]

Node mypeer0: Entries:

Node mypeer3: Entries:
  key = A9 , value = [( key = A9 , value = test)]

Node mypeer2: Entries:
  key = A9 , value = [( key = A9 , value = test)]

Node mypeer1: Entries:

Node mypeer4: Entries:

oc > entries -node mypeer5
Retrieving node mypeer5
Entries:
  key = A9 , value = [( key = A9 , value = test)]

oc >
\end{verbatim}
In this example it can be noted that data is replicated twice, as the 
property 
{\tt de.\-uniba.\-wiai.\-lspi.\-chord.\-service.\-impl.\-ChordImpl.\-successors} 
(see section \ref{config}) has been set to {\tt 2}. Therefore three peers 
have stored the data inserted in the previous example. 
The property 
{\tt de.\-uniba.\-wiai.\-lspi.\-chord.\-data.\-ID.\-number.\-of.\-displayed.\-bytes}
has been set to {\tt 1}. For this reason the first byte of the key of data item 
{\tt test} are displayed as a hexadecimal number. 
%
\item {\tt retrieve}: This command can be used to retrieve all data stored 
for a provided key within the DHT. For this purpose two parameters are required:  
{\tt node} which determines the peer, from that the retrieval is started, and 
{\tt key}, that defines the key to lookup. 

{\bf Example:}
\begin{verbatim}
oc > retrieve -node mypeer1 -key test
Values associated with key 'test':
test
Values retrieved from node 'mypeer1'
oc >
\end{verbatim}
%
\item {\tt remove}: This command removes previously inserted data from 
the Chord DHT and its parameters are the same as for the insert command. 

{\bf Example:}
\begin{verbatim}
oc > remove -node mypeer1 -key test -value test
Value 'test' with key 'test' removed successfully from node 'mypeer1'.
oc >
\end{verbatim}
After execution of this command, the command {\tt entries} should yield no 
peer, that holds any data. 
%
\item {\tt refs}: This command prints the reference a peer has in 
its finger table (\cite{stoica01chord}). It requires one parameter 
{\tt node}, that defines the name of the node, whose finger table 
is shown. 

{\bf Example:}
\begin{verbatim}
oc > refs -node mypeer3
Retrieving node mypeer3
Finger table:
  E5 (0-155)
  19 (156-157)
  21 (158)
  BE (159)

oc >
\end{verbatim}
This command prints the identifier of a peer in the finger table and 
behind an identifier a range of numbers. 
This range represents the indexes the node with that identifier has in 
the finger table.
%
\item {\tt successors}: This command prints out the successors 
(\cite{stoica01chord}) of a peer. It requires one parameter 
{\tt node}, that defines the name of the node, whose finger table 
is shown. 

{\bf Example:}
\begin{verbatim}
oc > successors -node mypeer1
Successor List:
  19
  21

oc >
\end{verbatim}
%
\item {\tt shutdown}: This command can be used to orderly shutdown 
a number of nodes of the Chord DHT, that then notify their predecessors and 
successors in the Chord ring, so that they can update their finger 
tables and successor lists. This command requires one parameter 
called {\tt names}, that defines a list of names of peers in a similar 
fashion as the {\tt create} command. This command has an optional 
parameter {\tt all}, which shuts down all peers. 

{\bf Example:}
\begin{verbatim}
oc > shutdown -names mypeer2
Node with name mypeer2 left.
oc >
\end{verbatim}
%
\item {\tt crash}: This command can be used to simulate the crash 
of a number of peers. These peers do not notify their predecessors
and successors. This command requires one parameter 
called {\tt names}, that defines a list of names of peers in a similar 
fashion as the {\tt create} command. This command also has an optional 
parameter {\tt all}, which crashes all peers. 

{\bf Example:}
\begin{verbatim}
oc > crash -names mypeer3
Crashing node mypeer3.
Node with name mypeer3 crashed.
oc >
\end{verbatim}
%
\item {\tt show}: This command displays the peers currently running 
in the order they are located on the Chord ring. It possesses an optional 
parameter {\tt count}, that requires no value and counts the 
number of peers currently running. 

{\bf Example:}
\begin{verbatim}
oc > show
Node list in the order as nodes are located on chord ring:
Node mypeer0 with id 19
Node mypeer4 with id 21
Node mypeer5 with id D9
Node mypeer1 with id E5
oc > show -count
No. of nodes currently running 4
oc > 
\end{verbatim}
\end{itemize}

\subsection{Connecting to and using a remote network}\label{remotecom}
The console can also be used to create a network with help of the 
{\tt ocsocket} communication protocol. For this purpose within one 
console a single Open Chord peer using the {\tt ocsocket} protocol 
is instantiated. The invocations (e.g. insert) are performed with help 
of the (synchronous) methods of {\tt Chord} interface. 

\begin{itemize}
%
\item {\tt joinN}: This command is used to create a single peer within 
a JVM. If no parameters are provided the peer creates a new network and 
listens on the standard port ({\tt 4242}) of Open Chord. 

Within one console assumed to run on a machine with IP-address 
{\tt 192.168.0.1} just type {\tt joinN} and hit enter. 
The following should be the result. 

{\bf Example:}
\begin{verbatim}
oc > joinN
Creating new chord overlay network!
URL of created chord node ocsocket://192.168.0.1/.
oc >
\end{verbatim}

In another console assumed to run on a machine with IP-address 
{\tt 192.168.0.2} just do the following. 

\begin{verbatim}
oc > joinN -port 8080 -bootstrap 192.168.0.1:4242
Trying to join chord network with boostrap URL ocsocket://192.168.0.1:4242/
URL of created chord node ocsocket://192.168.0.2:8080/.
oc >
\end{verbatim}

The {\tt port} parameter can be used to define a port different from the 
standard port. The {\tt bootstrap} parameter requires the host and the port 
of the bootstrap peer in the format shown above. 
%
\item {\tt insertN}: This command inserts a {\tt String} into the DHT. It requires 
two parameters: {\tt key} and {\tt value}. Both parameters are {\tt String}s and 
the first defines the key, with which the {\tt String} defined with help of the second, 
is inserted into the DHT. 

{\bf Example:}
\begin{verbatim}
oc > insertN -key test -value test
oc >
\end{verbatim}
%
\item {\tt retrieveN}: This command retrieves all data stored with 
the key provided with help of {\tt key} parameter. 

{\bf Example:}
\begin{verbatim}
oc > retrieveN -key test
Values associated with key 'test':
test
oc >
\end{verbatim}
%
\item {\tt removeN}: This command requires the same parameters as 
{\tt insertN} and removes the provided {\tt String} stored with 
the provided {\tt key}. 

{\bf Example:}
\begin{verbatim}
oc > removeN -key test -value test
Value 'test' with key 'test' removed.
oc >
\end{verbatim}
%
\item {\tt refsN}: This command shows the finger table, successor list, and 
predecessor of the Open Chord peer, which is running in the JVM of the console. 

{\bf Example:}
\begin{verbatim}
oc > refsN
Node: 0C
Finger table:
  7D (0-158)
Successor List:
  7D
Predecessor: 7D
oc >
\end{verbatim}
%
\item {\tt entriesN}: This command shows the data stored by the Open Chord peer, 
which is running in the JVM of the console. 

{\bf Example:}
\begin{verbatim}
oc > entriesN
Entries:
  key = A9 , value = [( key = A9 , value = test)]

oc >
\end{verbatim}
%
\item {\tt leaveN}: This command causes the peer to leave the DHT in an orderly 
fashion, so that it notifies its predecessor and successors about its departure. 

{\bf Example:}
\begin{verbatim}
oc > leaveN
Leaving network.
oc >
\end{verbatim}

\end{itemize}