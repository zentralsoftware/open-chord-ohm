\section{Using Open Chord}\label{usage}
After Open Chord has successfully been compiled and set up, it can be used from 
any Java application. This section describes the interfaces of Open Chord and how they 
can be used. For this purpose first a short overview of the architecture and the 
available communication protocols is provided. Afterward the interfaces relevant to 
an application programmer and their usage in typical use cases as e.g. creating and 
joining a network or inserting and retrieving data from the DHT are explained. 

\subsection{Architecture of Open Chord}
The architecture of Open Chord is divided into three layers. These layers are shown 
in figure \ref{architecture} below the layer that represents a Java application, that 
uses Open Chord. 
On the lowest layer the implementation of the employed communication 
protocol based on a network communication protocol (e.g. Java Sockets) is located. On 
top of this communication layer a communication abstraction layer resides, 
that provides interfaces, which abstract from the actually used 
communication protocol. 

\bildw{architecture}{architecture}{Architecture of Open Chord.}{13}

For this purpose two abstract classes have been developed, which represent the 
communication abstraction layer, and provide factory methods, to create instances of themselves for a specific communication protocol. The 
communication abstraction layer provides interfaces for synchronous communication 
between peers. Instances of class represent {\tt de.uniba.wiai.lspi.chord.com.Proxy} 
({\tt Proxy}) references to remote peers participating in an Open Chord overlay 
network. For each node in a Open Chord network an instance of 
{\tt de.uniba.wiai.lspi.chord.com.Endpoint} ({\tt Endpoint}) provides a connection 
point for remote peers with help of {\tt Proxy} for a 
specific communication protocol. The concrete implementations for a communication protocol are determined with help of the URL of a peer\footnote{This mapping from 
communication protocol (with help of URL) to the {\tt Proxy} of {\tt Endpoint} 
implementation is currently hard-coded within the corresponding factory methods}. 

Based on the communication abstraction layer the logic of the Chord overlay network 
such as how to find the successor of a peer\footnote{For details how the Chord 
DHT works refer to \cite{stoica01chord}.} has been implemented. 
This layer provides two interfaces to Java applications, which abstract from 
the implementation of routing within the Chord DHT. One interface 
{\tt de.\-uniba.\-wiai.\-lspi.\-chord.\-service.\-Chord} ({\tt Chord}) provides 
synchronous methods to retrieve, store, and remove values within the DHT. 
The other interface {\tt de.\-uniba.\-wiai.\-lspi.\-chord.\-service.\-AsynChord} 
({\tt AsynChord}) can be used for asynchronous retrieval, storage, and 
removal of data from the DHT. 
The Chord logic layer also is responsible for replication of data and maintenance 
of the properties that are necessary to keep the DHT running as described in 
\cite{stoica01chord}. This processes are transparent to the application using 
Open Chord, but must be configured as described in section \ref{config}. 

\subsection{Available communication protocols} 
Currently two communication protocol implementations are available for Open Chord: 
%
\begin{enumerate}
%
\item Local (within one JVM) communication protocol {\tt oclocal}: The implementation 
of this protocol (with help of sub-classes of {\tt Proxy} and {\tt Endpoint}) can be 
found in package {\tt de.\-uniba.\-wiai.\-lspi.\-chord.\-com.\-local}. This protocol has been developed in order to facilitate creation of an Open Chord network within a 
single JVM for testing purposes. 
%
\item Socket-based protocol {\tt ocsocket}: The implementation 
of this protocol (with help of sub-classes of {\tt Proxy} and {\tt Endpoint}) can be 
found in package {\tt de.\-uniba.\-wiai.\-lspi.\-chord.\-com.\-socket}. This protocol 
facilitates reliable communication between Open Chord peers based on TCP/IP sockets. 
%
\end{enumerate}
%
The protocol, which is used by an Open Chord peer is determined by its {\tt URL}. 
An implementation of {\tt URL} is provided in package {\tt de.\-uniba.\-wiai.\-lspi.\-chord.\-data}. This {\tt URL} currently provides some public 
constants, which can be used to create Strings representing an appropriate {\tt URL} 
for one of the protocols described above. The String array {\tt URL.KNOWN$\_$PROTOCOLS} 
contains the names of the two protocols and the integer constants 
{\tt URL.LOCAL$\_$PROTOCOL} and  {\tt URL.SOCKET$\_$PROTOCOL} represent their 
corresponding index\footnote{This should be changed in future releases to allow 
addition of custom communication protocols without change of implementation of 
{\tt URL} and factory methods of {\tt Proxy} and {\tt Endpoint}.}. So this can 
be used to create appropriate Strings representing an {\tt URL}, which can be 
used with the public constructor {\tt URL(String url)}. 
%
\subsection{The Chord and AsynChord interfaces}\label{interfaces} 
As presented before (figure \ref{architecture}) Open Chord provides two interfaces, 
which can be used by an application built on-top of Open Chord to retrieve, 
remove, and store data from/to the underlying DHT. These interfaces ({\tt Chord}, 
{\tt AsynChord}) provide some common methods that are important to create, join, and 
leave an Open Chord DHT. These methods are shown in listing \ref{common}. 

Each peer within an Open Chord DHT is represented by a unique identifier as proposed 
in \cite{stoica01chord}. This unique identifier is usually created with help of 
the {\tt URL} of a peer, when it creates or joins a network or the method, 
{\tt setURL(URL nodeURL)} is invoked. An identifier of a peer is represented by 
the class {\tt de.\-uniba.\-wiai.\-lspi.\-chord.\-data.\-ID}. An unique identifier 
of a peer can also be set by the application with help of the method {\tt setID(ID nodeID)}. Setting the {\tt URL} or {\tt ID} of a peer is only allowed {\bf before} 
an Open Chord network is joined or created. 

A new network can be created with help of the methods {\tt create()}, {\tt create(URL localURL)}, and {\tt create(URL localURL, ID localID)}. The first method can only be 
invoked if an {\tt URL} and an {\tt ID} is set for a peer. An {\tt ID} is automatically 
generated when an {\tt URL} is set. So when a custom {\tt ID} should be provided, the 
method {\tt setID(ID nodeID)} must be invoked {\bf after} invocation of 
{\tt setURL(URL nodeURL)}. The second {\tt create} method is a convenience method to 
avoid setting of an {\tt URL} before creation of a peer. The third {\tt create} method 
can be used to create a peer that listens on the provided {\tt URL localURL} and 
has the {\tt ID localID}.
%
\javafile{CommonMethods.java}{Common Methods.}{common}
%
The join methods allow a peer to join an existing 
Open Chord network. These methods correspond to the functionality of the three 
{\tt create} methods regarding the unique identifier of a peer, but need a further 
parameter, which is a {\tt URL} of a so-called bootstrap peer, that can be used 
by the peer, that wants to join, to get access to the existing DHT. How such a 
{\tt URL} can be found is out of scope of Open Chord and depends on the application, 
which is built on top of it. 

Although Open Chord handles crashes and desertion\footnote{A peer does not crash but leaves the network without notifying other peers} of peers transparently, it is 
recommended for a peer to announce to others, when it leaves the Open Chord network. 
For this purpose the two interfaces provide the method {\tt leave()}. By invocation 
of this method the peer makes sure that the routing tables of its neighbors immediately 
reflect the change within the network. 

In addition to these methods the {\tt Chord} interface provides the methods shown 
in listing \ref{chord}. They can be used to retrieve, insert, and remove 
data. As data is associated with a unique identifier within a DHT 
an instance of interface {\tt de.\-uniba.\-wiai.\-lspi.\-chord.\-service.\-Key} 
({\tt Key}) must be provided for these methods, 
with which the data to retrieve, insert, or remove can 
be associated. Any {\tt Serializable} Java object can be associated with a custom 
{\tt Key} implementation. 

\javafile{Chord.java}{The {\tt Chord} interface.}{chord}

A {\tt Key} implementation must just be able to create a 
byte representation of itself, which can be obtained by its {\tt getBytes()} method 
within a byte array. These bytes are used by Open Chord to create a hash value for 
a given data item with help of a hash function. As more than one instance may be 
associated with a single {\tt Key} an invocation of {\tt retrieve(Key key)} may return 
a {\tt Set} of associated instances. For {\tt Key } implementations and classes, whose 
instances should be stored in the DHT, it is strongly recommended (as with every class 
used in data structures or the Java Collections API) to overwrite the 
{\tt equals(Object)} and {\tt hashCode()} method of class {\tt Object} to ensure 
that Open Chord can correctly manage data storage. 

The method {\tt insert(Key key, Serializable entry)} inserts the provided instance 
{\tt entry} into the DHT associated with the given {\tt key}. This {\tt key} can 
later on be used to retrieve or remove {\tt entry}. In order to remove an {\tt entry}
the {\tt entry} and its {\tt key} must be known, as indicated by the method 
{\tt remove(Key key, Serializable entry)}. 

It is {\bf important} to note that, with the current implementation of Open Chord, 
class definitions (the byte code of classes), whose instances are used as {\tt Key} 
or as data associated with a {\tt Key} must be present on every peer within the 
network as Open Chord does not support class loading from remote peers. 

The methods of the {\tt Chord} interface are synchronous and block the thread, that 
invokes them, until the operation has been performed and a result has been obtained. 
This may not be desired in all cases, as invocation of the methods described above 
may take some time. While data is retrieved it may be possible that the same thread, 
which wants the data, wants to insert, remove or retrieve some other data in parallel. 
For this purpose Open Chord provides an interface with methods, that allow 
asynchronous processing of data retrieval, removal, and storage. 

These methods can be divided into two kinds. The first kind of methods presented in 
listing \ref{asynchord} in addition to the provided {\tt Key} or data requires an 
instance of interface 
{\tt de.\-uniba.\-wiai.\-lspi.\-chord.\-service.\-ChordCallback}, which is presented 
in listing \ref{callback}. 
The other kind of methods returns instances of 
{\tt de.\-uniba.\-wiai.\-lspi.\-chord.\-service.\-ChordFuture} 
(listing \ref{chordfuture}), which can be used later on to determine if an 
asynchronous invocation has been completed and to obtain its result 
(if necessary). 

\javafile{ChordCallback.java}{The {\tt ChordCallback} interface.}{callback}

The methods of the first kind require an instance of {\tt ChordCallback}, which 
is notified, when asynchronously performed retrieval, removal, or storage of data 
has been completed. These methods return directly after invocation and notification 
takes place with help of the corresponding callback method {\tt retrieved(...)}, 
{\tt removed(...)}, or {\tt inserted(...)}. 
To these method the affected {\tt Key} and {\tt entry} (if required) or result 
(in case of data retrieval) is provided. If the processing of retrieval, removal, 
or storage of data has failed a {\tt Throwable} that represents the cause of 
the failure is provided. This {\tt Throwable} is {\tt null} if invocation 
completes successfully. 

\javafile{AsynChord.java}{The {\tt AsynChord} interface.}{asynchord}

%HIER WEITER...
The second kind of methods does not require a callback object as a parameter, but 
instead immediately returns an instance of {\tt ChordFuture}. The methods to insert 
and remove data both return an instance of {\tt ChordFuture} which can be used 
to test if the corresponding invocation of {\tt insertAsync(...)} or {\tt removeAsync(...)} has been done with help of the {\tt isDone()} method. 
%
\javafile{ChordFuture.java}{The {\tt ChordFuture} interface.}{chordfuture}
%
The {\tt isDone()}  
method returns {\tt true} when the invocation has been performed successfully. 
When the invocation is still being performed {\tt false} is returned. When 
the invocation has been performed, but has failed, a {\tt ServiceException} 
is thrown by {\tt isDone()}. 
If a thread wants to wait until the operation associated with a {\tt ChordFuture} 
finishes, it can invoke {\tt waitForBeingDone()}, which blocks the calling thread 
until the end of the operation. This method throws a {\tt ServiceException} if the 
operation fails and an {\tt InterruptedException} occurs when the thread has been 
interrupted. When this method returns the associated operation has been performed. 

\javafile{ChordRetrievalFuture.java}{The {\tt ChordRetrievalFuture} interface.}{chordretrievalfuture}

The method to retrieve data from the DHT returns a {\tt ChordRetrievalFuture} presented 
in listing \ref{chordretrievalfuture}. This interface extends {\tt ChordFuture} by a method to obtain the result of the retrieval. This method behaves like 
{\tt waitForBeingDone()} regarding the declared exceptions, but returns the result 
as soon as the retrieval has been performed to the calling thread. If there is no 
data associated with the provided {\tt Key} an empty {\tt Set} is returned. 

\subsection{Examples}
\subsubsection{Creating a new Chord overlay network} 
In order to create a new network one of the {\tt create(...)} methods 
of the {\tt Chord} interface or {\tt AsynChord} interface has to be invoked 
on an instance of 
{\tt de.\-uniba.\-wiai.\-lspi.\-chord.\-service.\-impl.\-ChordImpl}. 

\javafile{CreateNetwork.java}{Creating an Open Chord network.}{create}

{\tt ChordImpl} implements both interfaces. An instance of it can be created with help of its public constructor. Listing \ref{create} shows an example for creation of a new 
network. For this purpose a {\tt URL} for the {\tt ocsocket} protocol is created. 
This {\tt URL} becomes the {\tt URL} of the Open Chord peer. It is recommended to 
automatically determine the host name and IP-address of a peer with help of 
{\tt java.net.InetAddress} and to use the hosts IP-address as the host part of 
the {\tt URL}. 

\subsubsection{Joining an existing Chord overlay network} 

Only the first peer of an Open Chord network can create a network. To add peers 
to an existing network one of the {\tt join(...)} methods of {\tt Chord} or 
{\tt AsynChord} has to be invoked. Listing \ref{join} shows how an existing 
network can be joined. 

\javafile{JoinNetwork.java}{Joining an Open Chord network.}{join}

This works similar to network creation, but additionally a {\tt URL} of 
a peer, which already is part of the network, that should be joined, is required. 
This peer is called a bootstrap peer and its {\tt URL} is called a bootstrap 
{\tt URL}. How this {\tt URL} is obtained depends on the application, that wants 
to use the DHT. Both {\tt URL}s used in listing \ref{join} must represent an 
{\tt URL} of the same protocol. 

Please note that the {\tt URL} of a peer must reflect the real host name of the host 
on which the peer is located. In this examples we used {\tt localhost}, so that the examples 
can only be executed on one machine. It is possible to provide the host name as a parameter 
to the main method or to determine it with help of the Java API ({\tt java.net.InetAddress}). 
It is also possible to create more than one chord peer (that uses the {\tt ocsocket} protocol) 
within one JVM. 

\subsubsection{Leaving a network}  
To leave a network the {\tt leave()} method of {\tt Chord} or {\tt AsynChord} has 
to be invoked. 

\subsubsection{Inserting and Removing data} 

In order to work with the DHT at least one implementation of the {\tt Key} 
interface must be provided. Nevertheless one instance of the DHT can be used 
with several {\tt Key} implementations. For the following examples a {\tt Key}
implementation (listing \ref{key}), which uses a {\tt String} as key, is used.

\javafile{StringKey.java}{A {\tt Key} implementation.}{key}

To synchronously insert data into the DHT a reference to the DHT of type 
{\tt Chord} is required. On this instance the insert method can be invoked 
as shown in listing \ref{insertsynch}. Asynchronous insertion of data 
can be performed with help of a reference to the DHT of type {\tt AsynChord}. 

\javafile{SynchronousInsert.java}{Inserting data.}{insertsynch}

Asynchronous use of the DHT is shown with help of removal of data, which is 
analogous to asynchronous insertion of data. Two possibilities exist for 
asynchronous used as described in section \ref{interfaces}. First removal of 
data using a {\tt ChordFuture} is shown in listing \ref{removeasynchf}. 

\javafile{AsynchronousRemoveFuture.java}{Removing data asynchronously.}{removeasynchf}

The method {\tt removeAsync(...)} in line 7 of listing \ref{removeasynchf} returns 
immediately and provides an instance of {\tt ChordFuture} to the calling thread. 
This instance ({\tt future}) can later on be used to test if the removal has finished, when it is necessary to know, that the removal has been performed. For this 
purpose in line 12 it is tested if this is the case. If it is not true, 
{\tt future.waitForBeingDone()} is invoked to block the thread until the removal 
has been performed. As the thread may be interrupted while waiting, a while loop 
(line 13) ensures that the thread really waits until the removal has finished. 
Insertion of data with this invocation model is performed analogously. 

Another possibility to remove or insert data is to use the corresponding method 
of {\tt AsynChord}, which requires an instance of {\tt ChordCallback} as 
a parameter. Listings \ref{mycallback} and \ref{callbackremove} show an 
implementation of {\tt ChordCallback} and how an invocation to 
remove data is made with this model. 

\javafile{MyCallback.java}{A sample callback implementation.}{mycallback}

\javafile{CallbackUsage.java}{Removing data asynchronously with help of a callback instance.}{callbackremove}

The {\tt ChordCallback} implementation just prints a message to the standard output 
of the JVM if removal of data is successfully performed. If removal fails, a message 
is printed to the error output of the JVM and the stack trace of the {\tt Throwable}, 
that caused the failure, is printed. In order to remove data the {\tt remove(...)} 
method of {\tt AsynChord} has to be invoked and provided with an instance of the 
callback. The method returns immediately and the removal is performed asynchronously. 
As soon as the removal has been performed, the {\tt removed(...)} method of 
{\tt MyCallback} is invoked and one of the messages described above is being printed. 
Insertion of data is processed analogously to removal of data. 

\subsubsection{Data retrieval} 
In order to retrieve data from the DHT there are also three possibilities as for 
insertion or removal of data. These work similar to the invocations described in 
the previous section. Therefore here no code samples are presented and the three 
possibilities are explained shortly. 
\begin{enumerate}
%
\item Synchronous retrieval with {\tt Chord.retrieve(Key key)}: An invocation of this 
method blocks the calling thread until the retrieval has been performed. A {\tt Set} 
of the results is returned to the calling thread. 
%
\item Asynchronous retrieval with 
{\tt AsynChord.retrieve(Key key, ChordCallback callback)}:\\ 
This method returns 
immediately and the result of the retrieval is provided to the callback instance 
passed to it through the method 
{\tt ChordCallback.retrieved(Key key, Set<Serializable> entries, Throwable t)}. 
The parameter {\tt key} is the key, for which data has been retrieved, {\tt entries} 
contains the results, and {\tt t} is {\tt null} if the retrieval is successful. 
When the retrieval has failed, {\tt t} is not {\tt null} and it is an instance of the exception, which has been raised by the retrieval. 
%
\item Asynchronous retrieval with {\tt AsynChord.retrieveAsync(Key key)}: \\
This method immediately returns an instance 
of {\tt ChordRetrievalFuture} which can be used to obtain the result of the retrieval 
with help of the method {\tt ChordRetrievalFuture.getResult()}. This method works similar 
to {\tt ChordFuture.waitForBeingDone()}, but returns a {\tt Set} (possibly empty if 
no data is associated with the key used for retrieval) as soon as the retrieval has 
finished. If retrieval fails, a {\tt ServiceException} is raised. As the thread may 
be interrupted, an {\tt InterruptedException} is thrown in that case. For this reason 
if the thread invoking {\tt getResult()} must wait until retrieval is finished, the 
invocation of {\tt getResult()} must be guarded by a while loop as described earlier 
for {\tt ChordFuture.waitForBeingDone()}. 
%
\end{enumerate}

