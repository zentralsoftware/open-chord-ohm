%===============================================================================
% zentrale Layout-Angaben und Befehle
%===============================================================================
%
% fr bessere Sicht von falschen Umbrchen die Option draft benutzen
% dadurch knnen aber die eingebundenen Bilder nicht sichtbar sein 
\documentclass[a4paper, 12pt]{article}
%
% hier zun�hst die bentigten packages
\usepackage[english]{babel}
\usepackage[latin1]{inputenc}
%\usepackage{fancyhdr}
\usepackage{verbatim}
\usepackage[T1]{fontenc}
\usepackage{ae}
%\usepackage{listings}
\usepackage{color}
%\usepackage{listings}
%\usepackage{bibgerm}
\usepackage{wrapfig}
%
% Einbindung des Grafik-Pakets
%\ifx\pdfoutput\undefined
\usepackage[dvips]{graphicx}
%\else
%\usepackage[pdftex]{graphicx}
%\pdfcompresslevel=9
%\pdfpageheight=297mm
%\pdfpagewidth=210mm
%\fi
%
% Page-Layout
\setlength\headheight{14pt}
\setlength\topmargin{-15,4mm}
\setlength\oddsidemargin{-0,4mm}
\setlength\evensidemargin{-0,4mm}
\setlength\textwidth{160mm}
\setlength\textheight{252mm}
%
% Absatzeinstellungen
\setlength\parindent{0mm}
\setlength\parskip{2ex}
%
% Kopf- und Fusszeile
%\pagestyle{fancy}
%\fancyhf{} % alles lschen
%\fancyhead[LO]{\footnotesize\sc\nouppercase{\leftmark}}
%\fancyhead[RO]{\footnotesize\sc\nouppercase{\rightmark}}
%\fancyfoot[LO]{\footnotesize\sc Distributed and Mobile Systems Group}
%\fancyfoot[RO]{\thepage}
%\renewcommand{\headrulewidth}{0pt}
%\renewcommand{\footrulewidth}{0pt}
%
% bessere Fehlermeldungen
\errorcontextlines=999
%
% Anweisung zur Erstellung der Titelseite
% #1 = Name der Diplomarbeit
% #2 = Autor
% #3 = Abgabedatum
% #4 = Autor
%\renewcommand{\maketitle}[4]

%\end{titlepage}
%}
%
% wird fr Hintergrund von Code bentigt
%\definecolor{hellgrau}{gray}{0.9}
%
% Einstellungen fr Java-Code
%\lstdefinestyle{javaStyle}{%
%  basicstyle=\small,%
%  backgroundcolor=\color{hellgrau},%
%  keywordstyle=\bfseries,%
%  showstringspaces=false,%
%  %language=Java,%
%  %numbers=left,%
%  numberstyle=\tiny,%
%  stepnumber=1,%
%  numbersep=5pt,%
%  extendedchars=true,%
%  xleftmargin=2em,%
%  lineskip=-1pt,%
%  breaklines%
%}
%
% neues environment fr Java-Sourcecode
% #1 = "caption={Hier eigene �erschrift}, label={Hier eigenes Label}"
%\lstnewenvironment{javacode}[1][]{%
%\lstset{style=javaStyle,#1}%
%}{}

\makeatletter
\def\mycounterlabel#1#2{% #1=label, #2=counter
    \begingroup % to keep change to \@currentlabel local
    %\def
    %\@currentlabel{\arabic{#2}}% label := the value of #2
    \label{#1}%
    \endgroup}
\makeatother

%
% Befehl zum Einbinden von Java-Sourcecode aus Datei
% #1 = Dateiname relativ zu src-Verzeichnis
% #2 = �erschrift
% #3 = Label
\newcounter{Listing}
\setcounter{Listing}{0}
\newcommand{\javafile}[3]{
  \refstepcounter{Listing}
  \begin{center}
    \mycounterlabel{#3}{Listing}Listing \theListing: #2 \\
  \end{center}
  \vspace{-1cm}
  \verbatiminput{src/#1}
}
%
% Einbindung eines Bildes
% #1 = label fr \ref-Verweise
% #2 = Name des Bildes ohne Endung relativ zu images-Verzeichnis
% #3 = Beschriftung
% #4 = Breite des Bildes im Dokument in cm
\newcommand{\bildw}[4]{%
  \begin{figure}[htb]%
    \begin{center}%
      \includegraphics[width=#4cm]{images/#2}%
      %\vskip -0.3cm%
      \caption{#3}%
      %\vskip -0,2cm%
      \label{#1}%
    \end{center}%
  \end{figure}%
}

%
% Einbindung eines Bildes mit Seitenbreite
% #1 = label f�r \ref-Verweise
% #2 = Name des Bildes ohne Endung relativ zu images-Verzeichnis
% #3 = Beschriftung
\newcommand{\bild}[3]{%
  \begin{figure}[htb]%
    \centering%
    \includegraphics[width=\textwidth]{images/#2}%
    \vskip -0.3cm%
    \caption{#3}%
    \vskip -0,2cm%
    \label{#1}%
\end{figure}%
}

%
% Umgebung fr Fliesstext um Grafik
% #1 = Ausrichtung: r, l, i, ...
% #2 = Breite des Bildes in cm
% #3 = Name des Bildes ohne Endung relativ zu images-Verzeichnis
% #4 = Beschriftung
% #5 = label fr \ref-Verweise
\newcommand{\fliesstext}[5]{%
\begin{wrapfigure}{#1}{#2cm}%
\includegraphics[width=#2cm]{images/#3}%
\caption{#4}%
\label{#5}%
\end{wrapfigure}%
}
%%% Local Variables: 
%%% mode: latex
%%% TeX-master: t
%%% End: 
